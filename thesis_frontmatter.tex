%
%UCSD Master's Thesis - Kevin Chau, Biology 2019
% ----------------------------------------------------------------------
%

\title{Isoform transcriptome of developing brain provides new insights into autism risk variants}

\author{Kevin Khai Chau}
\degreeyear{2019}

\degreetitle{Master of Science}

\field{Biology}

\chair{Lilia M. Iakoucheva}
\cochair{Scott Rifkin}

%  The rest of the committee members  must be alphabetized by last name.
\othermembers{
Barry Grant\\
}
\numberofmembers{3} % |chair| + |cochair| + |othermembers|


%% START THE FRONTMATTER
\begin{frontmatter}

%% TITLE PAGES
\makefrontmatter

%% SETUP THE TABLE OF CONTENTS
\tableofcontents
\listoffigures
\listofsuppfigures
\listoftables

%% ACKNOWLEDGEMENTS
\begin{acknowledgements}
I would like to acknowledge Professor Lilia M. Iakoucheva for her support as the chair of my committee. \par
I would also like to thank my post-doctoral mentor Dr. Pan Zhang for his guidance throughout this project, as well as Dr. Patricia M. Losada, Dr. Akula Bala Pramod, Dr. Jorge Urresti, and Dr. Megha Amar for all of their support. \par
This thesis, in full, is currently being prepared for submission for publication of the material. Chau, Kevin K.; Zhang, Pan; Urresti, Jorge; Amar, Megha; Pramod, Akula Bala; Corominas, Roser; Lin, Guan Ning; Iakoucheva, Lilia M. The thesis author was the primary investigator and author of this material. \par
\end{acknowledgements}


\begin{abstract}
Alternative splicing plays important role in brain development, however its global contribution to human neurodevelopmental diseases (NDD) has not been fully investigated. Here, we examined the relationship between splicing isoform expression and \textit{de novo} loss-of-function mutations implicated in autism. We constructed isoform transcriptome of the developing human brain, and observed better resolution and stronger signals at the isoform-level compared to the gene-level transcriptome. We identified differentially expressed isoforms and isoform co-expression modules enriched in autism loss-of-function mutations. These isoforms have higher prenatal expression, are enriched in microexons, and are co-expressed with a unique set of partners. We experimentally test the impact of splice site mutations in five NDD risk genes, including \textit{SCN2A}, \textit{DYRK1A} and \textit{BTRC}, and demonstrate exon skipping. Furthermore, our results suggest that the splice site mutation in \textit{BTRC} reduces its translational efficiency, likely impacting Wnt signaling through impaired degradation of $\beta$-catenin. We propose that functional effect of mutations associated with human diseases should be investigated at isoform- rather than gene-level resolution.

\end{abstract}


\end{frontmatter}
